\documentclass[a4paper,12pt]{article}

\usepackage{ctex}          % 中文支持
\usepackage{amsmath}       % 数学公式
\usepackage{graphicx}      % 图片支持
\usepackage{enumitem}      % 自定义列表

% 自动编号 Q1 / Q2 / Q3 ...
\newcounter{qcounter}
\newcommand{\question}[1]{%
	\stepcounter{qcounter}%
	\noindent\textbf{Q\theqcounter: #1}\par
}

\newcommand{\answer}[1]{%
	\noindent\textbf{A:} #1\par\vspace{1em}
}

% 定义环境(可选)
\newcommand{\definition}[1]{%
	\noindent\textbf{【定义】} #1\par\vspace{0.5em}
}

\begin{document}
	
	\section*{第一章笔记:TCP 三次握手}
	
	\question{为什么 TCP 需要三次握手?}
	\answer{因为两次握手无法确认双方都能可靠收发对方的报文。}
	
	\question{如果第三次握手的 ACK 丢失,会怎样?}
	\answer{服务器认为连接已建立,客户端会重发 ACK 或继续通信。}
	
	\definition{TCP(传输控制协议)是一种面向连接、可靠的传输层协议。}

	% 插入图片示例(可删)
	%\begin{center}
	%    \includegraphics[width=0.5\textwidth]{handshake.png}
	%    \par\textit{图1:三次握手示意图}
	%\end{center}
	
\end{document}
